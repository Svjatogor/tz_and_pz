\documentclass[a4paper]{G2-105}
\usepackage[utf8]{luainputenc}

\usepackage[dvipsnames]{xcolor}
\usepackage{listings}

\graphicspath{{figures/}}

\newcommand{\CommonSoftwareRequirements}{
\begin{itemize}
\item git;
\item nvm;
\item Node.js~v6.0.0;
\item npm~v3.8.6;
\item MongoDB~v2.6.12;
\item Stardog~v4.1.
\end{itemize}
}

\newcommand{\CommonBrowserRequirements}{
\begin{itemize}
\item Google Chrome~30.0;
\item Mozilla Firefox~25.0;
\item Apple Safari~6;
\item Microsoft Internet Explorer~9.
\end{itemize}
}

\newcommand{\CommonSystemRequirements}{
\begin{itemize}
\item процессор с тактовой частотой не ниже 2.7 ГГц;
\item объем оперативной памяти не ниже 512 Мб;
\item объем жесткого диска не ниже 10 Гб.
\end{itemize}
}

\newcommand{\CommonClientRequirements}{
\begin{itemize}
\item процессор с тактовой частотой не ниже 1600 МГц;
\item объем оперативной памяти не ниже 1024 Мб;
\item объем жесткого диска не ниже 20 Гб;
\item монитор с диагональю не менее 15 дюймов;
\item манипулятор типа «мышь», клавиатура.
\end{itemize}
}

\newcommand{\CommonGoal}{повышение качества принимаемых решений по управлению отходами на предприятии за счет организации интеллектуальной поддержки процесса принятия решений на основе онтологической модели представления знаний и логического вывода на онтологии}


\newcommand{\CommonGoalCap}{Повышение качества принимаемых решений по управлению отходами на предприятии за счет организации интеллектуальной поддержки процесса принятия решений на основе онтологической модели представления знаний и логического вывода на онтологии}

\newcommand{\CommonTasks}{
\begin{itemize}
\item провести анализ процесса и специфики управления отходами на предприятии с целью построения информационно-логической модели предметной области и формирования требований к модели представления знаний;
\item разработать концепцию поддержки принятия решений по управлению отходами предприятия на основе онтологической модели представления знаний и логического вывода на онтологии;
\item разработать онтологическую модель предметной области и алгоритм генерации стратегии управления отходами предприятия на основе логического вывода на онтологии;
\item разработать и протестировать интеллектуальную систему поддержки принятия решений по управлению отходами на основе предложенных модели и алгоритма.
\end{itemize}
}

\def\CommonOntologyModelFormula{O = \left<C,~R,~F\right >}
\def\CommonOntologyModelDescription{\begin{VSTUFormulaWhereList}
\item $C$~--- конечное множество понятий (концептов) предметной области, которую представляет онтология О;
\item $R$~--- конечное множество отношений между концептами;
\item $F$~--- конечное множество функций интерпретации, заданных на концептах и/или отношениях онтологии О.
\end{VSTUFormulaWhereList}
}

\newcommand{\CommonTprFormulaCompact}{$DS = \left<S,~A,~C,~M,~P,~R\right>$, где $S$~-- описание ситуации принятия решений, состоящее из множества численных и качественных параметров: $S = \left\{P_{q1},~P_{q2},~...,~P_{qn1},~P_{qn2}\right\}$; $A$~-- множество альтернатив, каждая из которых состоит из множества управляющих воздействий: $A = \left\{A_{c1},~A_{c2},~...,\right\}$; $C$~-- множество критериев, в виде качественных оценок ситуации с точки зрения предприятия; $M$~-- модель, позволяющая для каждой альтернативы рассчитать вектор критериев; $P$~-- система предпочтений для каждого из критериев; $R$~-- решающее правило выбора альтернативы}

\def\CommonTprFormula{DS = \left<S,~A,~C,~M,~P,~R\right>}
\def\CommonTprDescription{\begin{VSTUFormulaWhereList}
\item $S$~--- описание ситуации принятия решений, состоящее из множества численных и качественных параметров, $S = \left\{P_{q1},~P_{q2},~...,~P_{qn1},~P_{qn2}~\right\}$;
\item $A$~--- множество альтернатив, каждая из которых состоит из множества управляющих воздействий: $A = \left\{A_{c1},~A_{c2},~...,~A_{cm}\right\}$;
\item $C$~--- множество критериев, в виде качественных оценок ситуации с точки зрения предприятия;
\item $M$~--- модель, позволяющая для каждой альтернативы рассчитать вектор критериев;
\item $P$~--- система предпочтений для каждого из критериев;
\item $R$~--- решающее правило выбора альтернативы.
\end{VSTUFormulaWhereList}
}

\newcommand{\CommonMetaontologyModels}{
\begin{itemize}
\item онтология отходов, описывающая их свойства и классы опасности, а также негативное влияние, которые они оказывают на окружающую среду;
\item онтология субъектов, взаимодействующих с отходами (предприятие, полигон и т. д.);
\item онтология методов управления отходами, описывающая методы (переработка, утилизации, транспортировки и т.д.) и их стоимость, экологический вред, который также должен оплачиваться субъектом согласно закону РФ.
\end{itemize}
}

\newcommand{\CommonMetaontologyFormulaCompact}{$M = \left<O_{M},~C_{M},~Inst_{M},~R_{M},~I_{M}\right>$, 

где $M$ -- метаонтологическая модель предметной области; $O_{M} = \left\{O_{W},~O_{M},~O_{S}\right\}$~-- множество онтологических моделей, объединенных в метаонтологию, $O_{W}$~-- онтология отходов, $O_{M}$~-- онтология методов управления отходами,  $O_{S}$~-- онтология субъектов управления отходами; $C_{M}$~-- конечное множество концептов, $C_{M} = \varnothing$; $Inst_{M}$~-- конечное множество экземпляров классов, $Inst_{M} = \varnothing$; $R_{M} = \left\{has,~uses,~includes\right\}$~-- конечное множество отношений метаонтологии; $I_{M}$~-- множество правил интерпретации и ограничений, $I_{M} = \varnothing$}

\def\CommonMetaontologyFormula{M = \left<O_{M},~C_{M},~Inst_{M},~R_{M},~I_{M}\right>}
\def\CommonMetaontologyDescription{\begin{VSTUFormulaWhereList}
\item $M$~--- метаонтологическая модель предметной области;
\item $O_{M}$~--- множество онтологических моделей, объединенных в метаонтологию: $O_{M} = \left\{O_{W},~O_{M},~O_{S}\right\}$, где $O_{W}$~-- онтология отходов, $O_{M}$~-- онтология методов управления отходами,  $O_{S}$~-- онтология субъектов управления отходами;
\item $C_{M}$~--- конечное множество концептов, $C_{M} = \varnothing$;
\item $Inst_{M}$~--- конечное множество экземпляров классов, $Inst_{M} = \varnothing$;
\item $R_{M}$~--- конечное множество отношений метаонтологии $M$: $R = \left\{r_{M1},~r_{M2},~r_{M3}\right\}$, где $r_{M1}$~-- отношение has «имеет», $r_{M2}$~-- отношение uses «использует», $r_{M3}$~-- отношение includes «включает»;
\item $I_{M}$~--- множество правил интерпретации и ограничений, $I_{M} = \varnothing$.
\end{VSTUFormulaWhereList}
}

\newcommand{\CommonWasteOntologyCompact}{$O_{W} = \left<C_{W},~Inst_{W},~R_{W},~I_{W}\right>$, 

где $C_{W}$~-- конечное множество концептов онтологии отходов, $C_{W} = \left\{C_{W1},~...,~C_{W26}\right\}$; $Inst_{W}$~-- множество экземпляров классов онтологии отходов, $Inst_{W} = \left\{i_{W1},~i_{W2},~...,~i_{Wj},~...,~i_{Wn}\right\}$; $R_{W}$~-- конечное множество отношений онтологии отходов: $R_{W} = \left\{r_{W1},~...,~r_{W8}\right\}$, где $r_{W1}$~-- отношение $hasHazard$, $r_{W2}$~-- отношение $hasAggregateState$, $r_{W3}$~-- relation $hasOrigin$, $r_{W4}$~-- отношение $hasAmount$, $r_{W5}$~-- отношение $hasTitle$, $r_{W6}$~-- отношение $hasEcolTax$, $r_{W7}$~-- отношение $is$, $r_{W8}$~-- отношение $is-a$; $I_{W}$~-- множество правил}

\def\CommonWasteOntologyFormula{O_{W} = \left<C_{W},~Inst_{W},~R_{W},~I_{W}\right>}
\def\CommonWasteOntologyDescription{\begin{VSTUFormulaWhereList}
\item $C_{W}$~--- конечное множество концептов онтологии отходов: $C_{W} = \left\{C_{W1},~...,~C_{W26}\right\}$;
\item $Inst_{W}$~--- множество экземпляров классов онтологии отходов: $Inst_{W} = \left\{i_{W1},~i_{W2},~...,~i_{Wj},~...,~i_{Wn}\right\}$;
\item $R_{W}$~--- конечное множество отношений онтологии отходов: $R_{W} = \left\{r_{W1},~...,~r_{W8}\right \}$, где $r_{W1}$~-- отношение $hasHazard$, $r_{W2}$~-- отношение $hasAggregateState$, $r_{W3}$~-- relation $hasOrigin$, $r_{W4}$~-- отношение $hasAmount$, $r_{W5}$~-- отношение $hasTitle$, $r_{W6}$~-- отношение $hasEcolTax$, $r_{W7}$~-- отношение $is$, $r_{W8}$~-- отношение $is-a$;
\item $I_{W}$~--- множество правил и ограничений (методика их добавления рассмотрена далее).
\end{VSTUFormulaWhereList}
}

\newcommand{\CommonMethodOntologyCompact}{$O_{M} = \left<C_{M},~Inst_{M},~R_{M},~I_{M}~\right>$, где $C_{M}$~-- конечное множество концептов онтологии методов управления отходами, $C_{M} = \left\{C_{M1},~...,~C_{M6}\right\}$; $Inst_{M}$~-- множество экземпляров классов онтологии методов управления отходами, $Inst_{M} = \left\{i_{M1},~i_{M2},~...,~i_{Mj},~...,~i_{Mn}\right\}$; $R_{M}$~-- множество отношений онтологии методов управления отходами, $R_{M} = \left\{r_{M1},~...,~r_{M8}\right\}$, где $r_{M1}$~-- отношение $is$, $r_{M2}$~-- отношение $hasTitle$, $r_{M3}$~-- отношение $hasMethod$, $r_{M4}$~-- отношение $processedBy$, $r_{M5}$~-- отношение $hasHarmfulEffect$, $r_{M6}$~-- отношение $hasCostOnDistance$, $r_{M7}$~-- отношение $hasCostOnWeight$, $r_{M8}$~-- отношение $hasCostByService$; $I_{M}$~-- множество правил интерпретации и ограничений, $I_{M} = \varnothing$}

\def\CommonMethodOntologyFormula{O_{M} = \left<C_{M},~Inst_{M},~R_{M},~I_{M}\right>}
\def\CommonMethodOntologyDescription{\begin{VSTUFormulaWhereList}
\item $C_{M}$~--- конечное множество концептов онтологии методов управления отходами: $C_{M} = \left\{C_{M1},~...,~C_{M6}\right\}$;
\item $Inst_{M}$~--- множество экземпляров классов онтологии методов управления отходами: $Inst_{M} = \left\{i_{M1},~i_{M2},~...,~i_{Mj},~...,~i_{Mn}\right\}$;
\item $R_{M}$~--- множество отношений онтологии методов управления отходами: $R_{M} = \left\{r_{M1},~...,~r_{M8}\right\}$, где $r_{M1}$~-- отношение $is$, $r_{M2}$~-- отношение $hasTitle$, $r_{M3}$~-- отношение $hasMethod$, $r_{M4}$~-- отношение $processedBy$, $r_{M5}$~-- отношение $hasHarmfulEffect$, $r_{M6}$~-- отношение $hasCostOnDistance$, $r_{M7}$~-- отношение $hasCostOnWeight$, $r_{M8}$~-- отношение $hasCostByService$;
\item $I_{M}$~--- множество правил интерпретации и ограничений, $I_{M} = \varnothing$.
\end{VSTUFormulaWhereList}
}

\newcommand{\CommonSubjectOntologyCompact}{$O_{S} = \left<C_{S},~Inst_{S},~R_{S},~I_{S}\right>$, где $C_{S}$~-- конечное множество концептов субъектов управления отходами, $C_{S} = \left\{C_{S1},~...,~C_{S6}\right\}$; $Inst_{S}$~-- множество экземпляров классов онтологии субъектов управления отходами, $Inst_{S} = \left\{i_{S1},~i_{S2},~...,~i_{Sj},~...,~i_{Sn}\right\}$; $R_{S}$~-- множество отношений онтологии субъектов управления отходами, $R_{S} = \left\{r_{S1},~...,~r_{S9}\right\}$, $r_{S1}$~-- отношение is, $r_{S2}$~-- отношение locatedIn, $r_{S3}$~-- отношение hasCoordinates, $r_{S4}$~-- отношение hasEcolOfGround, $r_{S5}$~-- отношение hasTitle, $r_{S6}$~-- отношение hasWaste, $r_{S7}$~-- отношение hasMethod, $r_{S8}$~-- отношение hasBudget, $r_{S9}$~-- отношение hasEcolOfAir; $I_{S}$~-- множество правил интерпретации и ограничений, $I_{S} = \varnothing$}

\def\CommonSubjectOntologyFormula{O_{S} = \left<C_{S},~Inst_{S},~R_{S},~I_{S}\right>}
\def\CommonSubjectOntologyDescription{\begin{VSTUFormulaWhereList}
\item $C_{S}$~--- конечное множество концептов субъектов управления отходами: $C_{S} = \left\{C_{S1},~...,~C_{S6}\right\}$;
\item $Inst_{S}$~--- множество экземпляров классов онтологии субъектов управления отходами: $Inst_{S} = \left\{i_{S1},~i_{S2},~...,~i_{Sj},~...,~i_{Sn}\right\}$;
\item $R_{S}$~--- множество отношений онтологии субъектов управления отходами: $R_{S} = \left\{r_{S1},~...,~r_{S9}\right\}$, где $r_{S1}$~-- отношение is, $r_{S2}$~-- отношение locatedIn, $r_{S3}$~-- отношение hasCoordinates, $r_{S4}$~-- отношение hasEcolOfGround, $r_{S5}$~-- отношение hasTitle, $r_{S6}$~-- отношение hasWaste, $r_{S7}$~-- отношение hasMethod, $r_{S8}$~-- отношение hasBudget, $r_{S9}$~-- отношение hasEcolOfAir;
\item $I_{S}$~--- множество правил интерпретации и ограничений, $I_{S} = \varnothing$.
\end{VSTUFormulaWhereList}
}

\def\CommonTaskGivenFormula{s = \left<Coord_{s},~L_{s},~B_{s},~W_{s},~M_{s}\right>}
\def\CommonTaskGivenDescription{\begin{VSTUFormulaWhereList}
\item $s$~--- предприятие, являющиеся экземпляром класа $Subject$ онтологии $O_{S}$;
\item $Coord_{s}$~--- координаты предприятия, строка;
\item $L_{s}$~--- локация предприятия, определяющая город, регион и др.: $L_{s} = \left\{l_{1},~l_{2},~...,~l_{i},~...,~l_{n}\right\}$, где $l_{i}$~-- экземпляр класса $Subject$ онтологии $O_{S}$;
\item $B_{s}$~--- бюджет предприятия, число;
\item $W_{s}$~--- отходы предприятия, сгрупированные по методу управления отходами $W_{s} = \left\{W_{1},~W_{2},~...,~W_{i},~...,~W_{n}\right\}$, где $W_{i}$~-- подмножество отходов, объединенных общим признаком -- метод управления $m_{i}$, $W_{i} = \left\{w_{i1},~w_{i2},~...,~w_{ij}\right\}$, где $w_{ij}$~-- экземпляр класса $Waste$ онтологии $O_{W}$;
\item $M_{s}$~--- методы управления отходами предприятия: $M_{s} = \left\{m_{1},~m_{2},~...,~m_{i},~...,~m_{n}\right\}$, где $m_{i}$~--- экземпляр класса $Method$ онтологии $O_{M}$. 
\end{VSTUFormulaWhereList}
}

\def\CommonFormalOntologyTask{M|_{min(ecol, econ)} = \{M_{1},~M_{2}~...,~M_{k}\} : k = |W_{s}|.}
\def\CommonFormalOntologySolution{St_{s} = \{\left<W_{1},~M_{1}\right>,~\left<W_{2},~M_{2}\right>,~...,~\left<W_{n},~M_{n}\right>\}.}

\newcommand{\CommonScientificNovations}{
\begin{itemize}
\item Разработана интегрированная онтологическая модель представления знаний по управлению отходами предприятия, которая отличается от известных возможностью описания данных и знаний об объектах и субъектах процесса управления отходами на общем домене концептов, а также позволяет реализовывать логический вывод на онтологии на основе семантических запросов.
\item Разработан алгоритм генерации эффективной стратегии управления отходами на основе логического вывода на онтологической модели с использованием семантических запросов.
\end{itemize}
}

\newcommand{\CommonPracticalValue}{
\begin{itemize}
\item Разработанные в диссертационной работе модели и алгоритм позволяют производить генерацию эффективной стратегии управления отходами на предприятии. Реализованная система поддержки принятия решений включает в себя модуль генерации стратегии управления отходами на предприятии, а также онтологическую базу знаний отходов, методов и субъектов управления отходами. Разработана методика создания и расширения онтологической базы знаний предметной области, что позволяет применять данную систему, учитывая особенности различных субъектов управления отходами и видов отходов. В результате повышается качество принимаемых решений в области обращения с отходами на предприятии при использовании предложенной системой стратегии управления отходами.
\item Реализованная система поддержки принятия решений прошла аппробацию в учреждении ГБУЗ «Николаевское ЦРБ» в процессе обращения с отходами. 
\item Работа выполнена при поддержке РФФИ в рамках проекта №15-07-03541 «Интеллектуальная поддержка принятия решений по управлению сложными системами на основе интеграции различных типов рассуждений на знаниях, представленных онтологической моделью».
\end{itemize}
}

\newcommand{\CommonResults}{
\begin{itemize}
\item Проведен анализ процессов управления отходами на предприятии; информационных систем, используемых при принятии решений по управлению отходами; обзор моделей и методов, используемых при поддержке принятия решений по управлению отходами. Основным недостатком существующего процесса является сложность и трудоемкость процесса принятия верного решения экпертом в области управления отходами на предприятии. Построена информационно-логическая модель предметной области, включающая функциональную и объектную модели. Выявлены требования к модели представления знаний для описания объектов и субъектов процесса управления отходами на предприятии.
\item Разработана концепция поиска эффективной стратегии управления отходами на предприятии. Предложенная концепция предполагает создание автоматизированной системы для решения задачи генерации стратегии управления отходами, что позволяет сократить трудоемкость решения задачи и повысить обоснованность принимаемых решений, за счет применения моделей и методов искусственного интеллекта -- онтологической модели представления знаний и логического вывода на онтологиях.
\item Разработана интегрированная онтологическая модель представления знаний предметной области, состоящая из следующих компонентов, объединенных метаонтологией: (1) онтология отходов; (2) онтология методов управления отходами; (3) онтология субъектов управления отходами. Разработанная модель позволяет описывать объекты и субъекты процесса управления отходами на общем домене концептов и решать задачу поиска эффективной стратегии управления отходами посредством логического вывода на онтологии. Разработан алгоритм генерации эффективной стратегии управления отходами на предприятии на основе логического вывода на онтологической модели с использованием языка семантических запросов.
\item Разработана архитектура и реализована интеллектуальная система поддержки принятия решений для генерации эффективной стратегии управления отходами на предприятии на основе описанных моделей и алгоритма. Проведено тестирование системы и проверка соответствия полученой стратегии управления отходами с данными по учреждению ГБУЗ «Николаевское ЦРБ». Стратегия управления отходами, полученная в результате генерации системой, соответствует сформулированным критериям эффективности и соответствует выбранным методом управления отходами, примененным учреждением ГБУЗ «Николаевское ЦРБ». Данный результат позволяет сделать вывод об эффективности разработанных моделей и алгоритма.
\end{itemize}
}

\newcommand{\CommonKeywords}{Ключевые слова: управление отходами, поддержка принятия решений, метаонтология, OWL-DL, логический вывод на онтологии, RDF, SPARQL, интеллектуальная система поддержки принятия решений}
\newcommand{\CommonKeywordsEng}{Keywords: waste management, decision support, Metaontology, OWL-DL, inference logical consequences, RDF, SPARQL, Intelligent Decision Support System}

\newcommand{\CommonPublicationsVAK}{
\begin{enumerate}
\item Кульцова,~М.Б. Интеллектуальная поддержка принятия решений по управлению отходами на городских территориях на основе онтологической модели представления знаний~/ Кульцова~М.Б., Руднев~Р.Ю., Жукова~И.Г., Аникин~А.В.~//~Изв. ВолгГТУ. Серия «Актуальные проблемы управления, вычислительной техники и информатики в технических системах». Вып. 13~:~межвуз. сб. науч. ст.~/~ВолгГТУ.~--~Волгоград,~2015.~--~№ 13 (117).~-- C.~104-109.
\end{enumerate}
}

\newcommand{\CommonPublicationsOther}{
\begin{enumerate}
\setcounter{enumi}{1}

\item Руднев,~Р.Ю. Онтологический подход к поддержке принятия решений по управлению отходами на городских территориях~/~Руднев~Р.Ю., Кульцова~М.Б., Жукова~И.Г.~//~XII Международная научно-практическая конференция «Инновации на основе информационных и коммуникационных технологий» ИНФО-2015 (Сочи, 1-10 окт. 2015 г.)~:~сб. науч. ст.~--~Сочи, 2015.~--~С.~568-571.

\item Руднев,~Р.Ю. Концепция поддержки принятия решений по управлению отходами на городских территориях на основе онтологического и имитационного моделирования~/~Руднев~Р.Ю., Кульцова~М.Б.~//~XX региональная конференция молодых исследователей Волгоградской области (Волгоград, 10-13 нояб. 2015 г.)~:~тез. докл.~/~отв. ред. А.В. Навроцкий~;~Волгогр. гос. техн. ун-т [и др.].~--~Волгоград, 2016.~--~C.~143-144.

\item Kultsova~M. An ontology-based approach to intelligent support of decision making in waste management~/~Kultsova~M., Rudnev~R., Anikin~A., Zhukova~I.~//~CIT DS Creativity in Intelligent Technologies Data Science, The 7th International Conference on Information, Intelligence, Systems and Applications,~IISA2016,~Greece,~2016 (принята к публикации).

\end{enumerate}
}

\newcommand{\CommonPublications}{
\begin{enumerate}

\item [1] Кульцова,~М.Б. Интеллектуальная поддержка принятия решений по управлению отходами на городских территориях на основе онтологической модели представления знаний~/ Кульцова~М.Б., Руднев~Р.Ю., Жукова~И.Г., Аникин~А.В.~//~Изв. ВолгГТУ. Серия «Актуальные проблемы управления, вычислительной техники и информатики в технических системах». Вып. 13~:~межвуз. сб. науч. ст.~/~ВолгГТУ.~--~Волгоград,~2015.~--~№ 13 (117).~-- C.~104-109.

\item [2] Руднев,~Р.Ю. Онтологический подход к поддержке принятия решений по управлению отходами на городских территориях~/~Руднев~Р.Ю., Кульцова~М.Б., Жукова~И.Г.~//~XII Международная научно-практическая конференция «Инновации на основе информационных и коммуникационных технологий» ИНФО-2015 (Сочи, 1-10 окт. 2015 г.)~:~сб. науч. ст.~--~Сочи, 2015.~--~С.~568-571.

\item [3] Руднев,~Р.Ю. Концепция поддержки принятия решений по управлению отходами на городских территориях на основе онтологического и имитационного моделирования~/~Руднев~Р.Ю., Кульцова~М.Б.~//~XX региональная конференция молодых исследователей Волгоградской области (Волгоград, 10-13 нояб. 2015 г.)~:~тез. докл.~/~отв. ред. А.В. Навроцкий~;~Волгогр. гос. техн. ун-т [и др.].~--~Волгоград, 2016.~--~C.~143-144.

\item [4] Kultsova~M. An ontology-based approach to intelligent support of decision making in waste management~/~Kultsova~M., Rudnev~R., Anikin~A., Zhukova~I.~//~CIT DS Creativity in Intelligent Technologies Data Science, The 7th International Conference on Information, Intelligence, Systems and Applications,~IISA2016,~Greece,~2016.

\end{enumerate}
}

% Информация для генерации титульных листов, хедеров и футеров
\VSTUSetDocumentNumbersPrefix{}
\VSTUSetDocumentCode{МД-40-461-806-10.19-09.04.04-08-16}
\VSTUSetDocumentTypeDative{магистерской диссертации}
\VSTUSetDocumentTypeAccusative{магистерскую диссертацию}
\VSTUSetDocumentTypeEng{master's thesis}
\VSTUSetInitialData{задание, выданное научным руководителем кафедры ПОАС, утвержденное приказом ректора университета}
\VSTUSetOrder{1551--ст}{20}{октябрь}{2014}
\VSTUSetDeadline{01}{июня}
\VSTUSetFaculty{Электроники и вычислительной техники}
\VSTUSetDepartment{Программное обеспечение автоматизированных систем}
\VSTUSetDepartmentCode{10.19}
\VSTUSetDirection{09.04.04 Программная инженерия}
\VSTUSetHeadOfDepartment{Зав. кафедрой ПОАС}{д.т.н., проф.}{А. М. Дворянкин}{Дворянкин Александр Михайлович}
\VSTUSetDirector{доц. каф. ПОАС}{к.т.н.}{М. Б. Кульцова}{Кульцова Марина Борисовна}
\VSTUSetStandardsAdviser{ст. преп. каф. ПОАС}{}{О. Н. Ляпина}{Ляпина Ольга Николаевна}
\VSTUSetReviewer{доц. каф. САПР и ПК}{д.т.н., проф.}{Н. П. Садовникова}{Садовникова Наталья Петровна}
\VSTUSetStudent{ПРИН-2Н}{Р. Ю. Руднев}{Руднев Роман Юрьевич}{Руднева Романа Юрьевича}
\VSTUSetStudentFullNameEng{Roman Rudnev}
\VSTUSetTitle{Интеллектуальная система поддержки принятия  решений  по  управлению  отходами}
\VSTUSetTitleEng{Intelligent decision support system of waste management}

% Определние цветов
\definecolor{dimgray}{HTML}{696969}
\definecolor{silver}{HTML}{C0C0C0}
\definecolor{darkgray}{HTML}{A9A9A9}

% Определение стилей
\lstdefinestyle{appendix}{
xleftmargin=8mm,
captionpos=b,
basicstyle=\ttfamily\footnotesize,
breakatwhitespace=false, 
breaklines=true,
keepspaces=true,
showspaces=false,
showstringspaces=false,
showtabs=false,                  
tabsize=2,
frame=single,
numbers=left,
stepnumber=1,
numbersep=7pt,
numberstyle=\footnotesize\color{silver}
}

\lstdefinestyle{default}{
basicstyle=\ttfamily\small,
xleftmargin=8mm,
breakatwhitespace=false,         
breaklines=true,                 
captionpos=b,                    
keepspaces=true,
showspaces=false,                
showstringspaces=false,
showtabs=false,                  
tabsize=2,
}

% Определение поддержки формата Turtle
\lstdefinelanguage{ttl}{
sensitive=true,
morecomment=[l][\color{dimgray}]{@},
morecomment=[l][\itshape\color{darkgray}]{\#},
morestring=[b]",
morekeywords={rdf,rdfs,owl,xsd}
}


\begin{document}

%
\VSTUSetDocumentNumbersPrefix{Ж.}
\VSTUInitializeRD
%

\abstract{Аннотация}

Документ представляет собой описание рабочей документации к интеллектуальной системе поддержки принятия решений по управлению отходами на основе онтологической модели представления знаний и логического вывода на онтологиях. Приводится структура исходных файлов системы, описание классов системы, тестовый пример функционирования системы. В приложениях содержатся руководства пользователя и системного администратора.

Документ включает в себя страниц~---~\totalpages, таблиц~---~\totaltables, приложений~---~\totalappendices.

\CommonKeywords.

\newpage

\tableofcontents

\newpage

\chapter{Список исходных файлов}

В таблице~\ref{tab:source_files_server} приведен перечень файлов, необходимых для работы серверной части системы, исключая дополнительные (сторонние библиотеки).

\begin{longtable}[l]{|p{5cm}|p{3cm}|p{6.7cm}|}
\caption{Список файлов, необходимых для работы серверной части системы}
\label{tab:source_files_server}\tabularnewline
\hline
Имя файла & Путь к файлу & Описание \tabularnewline
\endfirsthead
\hline
Имя файла & Путь к файлу & Описание \tabularnewline
\endhead
\hline
auth.controller.js & api/controllers & Контроллер аутентификации \tabularnewline
\hline
method.controller.js & api/controllers & Контроллер способов управления отходами \tabularnewline
\hline
subject.controller.js & api/controllers & Контроллер субъектов управления отходами \tabularnewline
\hline
waste.controller.js & api/controllers & Контроллер отходов \tabularnewline
\hline
app.js & api/core & Express приложение \tabularnewline
\hline
config.js & api/core & Конфигурационный файл сервера \tabularnewline
\hline
passport.js & api/core & Настройка аутентификации \tabularnewline
\hline
roles.js & api/core & Настройка авторизации на основе ролей \tabularnewline
\hline
stardog.js & api/core & Надстройка над библиотекой Stardog \tabularnewline
\hline
translations.js & api/core & Настройка интернационализации \tabularnewline
\hline
en.json & api/locales & Словарь сообщений на английском языке \tabularnewline
\hline
ru.json & api/locales & Словарь сообщений на русском языке \tabularnewline
\hline
counter.model.js & api/models & Модель глобального счетчика \tabularnewline
\hline
user.model.js & api/models & Модель пользователя \tabularnewline
\hline
rdfsbase.storage.js & api/models & Базовый класс для всех хранилищ \tabularnewline
\hline
method.storage.js & api/models & Хранилище способов управления отходами \tabularnewline
\hline
subject.storage.js & api/models & Хранилище субъектов управления отходами \tabularnewline
\hline
waste.storage.js & api/models & Хранилище отходов \tabularnewline
\hline
base-ontology.ttl & api/owl & Базовая онтология \tabularnewline
\hline
test-ontology.ttl & api/owl & Тестовая онтология \tabularnewline
\hline
prod-ontology.ttl & api/owl & Онтология для релиза \tabularnewline
\hline
gbuz-ontology.ttl & api/owl & Онтология учреждения ГБУЗ Николаевское ЦРБ \tabularnewline
\hline
auth.routes & api/routes & Маршруты для аутентификации \tabularnewline
\hline
method.routes & api/routes & Маршруты для способов управления отходами \tabularnewline
\hline
subject.routes & api/routes & Маршруты для субъектов управления отходами \tabularnewline
\hline
waste.routes & api/routes & Маршруты для отходов \tabularnewline
\hline
index.js & api/routes & Конфигурация маршрутов \tabularnewline
\hline
common.js & api/tests & Общий настройки для тестирования \tabularnewline
\hline
auth.spec.js & api/tests/api & Набор тестов для проверки API аутентификации \tabularnewline
\hline
method.spec.js & api/tests/api & Набор тестов для проверки API способов управления отходами \tabularnewline
\hline
subject.spec.js & api/tests/api & Набор тестов для проверки API субъектов управления отходами \tabularnewline
\hline
waste.spec.js & api/tests/api & Набор тестов для проверки API отходов \tabularnewline
\hline
method.storage.spec.js & api/tests/models & Набор тестов для проверки хранилища способов управления отходами \tabularnewline
\hline
subject.storage.spec.js & api/tests/models & Набор тестов для проверки хранилища субъектов управления отходами \tabularnewline
\hline
waste.storage.spec.js & api/tests/models & Набор тестов для проверки хранилища отходов \tabularnewline
\hline
geoUtils.spec.js & api/tests/util & Набор тестов для проверки гео утилит \tabularnewline
\hline
owlUtils.spec.js & api/tests/util & Набор тестов для проверки owl утилит \tabularnewline
\hline
utils.spec.js & api/tests/util & Набор тестов для проверки утилит \tabularnewline
\hline
expressUtils.js & api/util & Набор функций для express приложения \tabularnewline
\hline
geoUtils.js & api/util & Набор функций для определения геокоординат \tabularnewline
\hline
owlUtils.js & api/util & Набор функций для написания owl запросов \tabularnewline
\hline
utils.js & api/util & Набор вспомогательных функций \tabularnewline
\hline
bootstrap.js & api & Стартовый файл сервера \tabularnewline
\hline
dummy.data.js & api & Стартовые данные для базы данных \tabularnewline
\end{longtable}

В таблице~\ref{tab:source_files_client} приведен перечень файлов, необходимых для работы клиентской части системы, исключая дополнительные (сторонние библиотеки и картинки).

\begin{longtable}[l]{|p{4cm}|p{5cm}|p{5.7cm}|}
\caption{Список файлов, необходимых для работы клиентской части системы}
\label{tab:source_files_client}\tabularnewline
\hline
Имя файла & Путь к файлу & Описание \tabularnewline
\endfirsthead
\hline
Имя файла & Путь к файлу & Описание \tabularnewline
\endhead
\hline
AppContainer.js & app/components & Компонент контейнера приложения \tabularnewline
\hline
CompanyForm.js & app/components & Компонент формы редактирования компании \tabularnewline
\hline
CompanyProfile.js & app/components & Компонент профиля компании \tabularnewline
\hline
GridContainer.js & app/components & Компонент контейнера таблицы \tabularnewline
\hline
Login.js & app/components & Компонент формы входа в систему \tabularnewline
\hline
MethodForm.js & app/components & Компонент формы редактирования способа управления отходами \tabularnewline
\hline
MethodTypeForm.js & app/components & Компонент формы редактирования вида управления отходами \tabularnewline
\hline
NavBreadcrumb.js & app/components & Компонент отображения навигационной цепочки \tabularnewline
\hline
NavLink.js & app/components & Компонент навигационной ссылки \tabularnewline
\hline
NavMenuItem.js & app/components & Компонент навигационного меню \tabularnewline
\hline
NavSidebar.js & app/components & Компонент навигационной панели \tabularnewline
\hline
UserForm.js & app/components & Компонент формы редактирования пользователя \tabularnewline
\hline
WasteForm.js & app/components & Компонент формы редактирования отходов \tabularnewline
\hline
WasteTypeForm.js & app/components & Компонент формы редактирования вида отходов \tabularnewline
\hline
app.js & app/core & Приложение \tabularnewline
\hline
Entity.js & app/core & Класс сущности ресурса \tabularnewline
\hline
errorHandler.js & app/core & Обработчик ошибок \tabularnewline
\hline
router.js & app/core & Роутер \tabularnewline
\hline
routes.js & app/core & Все маршруты приложения \tabularnewline
\hline
session.js & app/core & Пользовательская сессия \tabularnewline
\hline
StateRouter.js & app/core & Базовый класс роутера-состояний \tabularnewline
\hline
AggregateState.js & app/entities & Класс сущности агрегатного состояния \tabularnewline
\hline
Company.js & app/entities & Класс сущности компании \tabularnewline
\hline
HazardClass.js & app/entities & Класс сущности класса опасности \tabularnewline
\hline
Method.js & app/entities & Класс сущности способа управления отходами \tabularnewline
\hline
MethodType.js & app/entities & Класс сущности вида управления отходами \tabularnewline
\hline
Origin.js & app/entities & Класс сущности происхождения отходов \tabularnewline
\hline
User.js & app/entities & Класс пользователя \tabularnewline
\hline
Waste.js & app/entities & Класс сущности отходов \tabularnewline
\hline
WasteType.js & app/entities & Класс сущности вида отходов \tabularnewline
\hline
CreateRoute.js & app/routes/company & Класс маршрута для создания предприятия \tabularnewline
\hline
EditRoute.js & app/routes/company & Класс маршрута для редактирования предприятия \tabularnewline
\hline
IndexRoute.js & app/routes/company & Класс маршрута для просмотра всех предприятий \tabularnewline
\hline
ShowRoute.js & app/routes/company & Класс маршрута для просмотра предприятия \tabularnewline
\hline
LoginRoute.js & app/routes/login & Класс маршрута для входа в систему \tabularnewline
\hline
LogoutRoute.js & app/routes/logout & Класс маршрута для выхода из системы \tabularnewline
\hline
CreateRoute.js & app/routes/method & Класс маршрута для создания способа управления отходами \tabularnewline
\hline
EditRoute.js & app/routes/method & Класс маршрута для редактирования способа управления отходами \tabularnewline
\hline
ShowRoute.js & app/routes/method & Класс маршрута для просмотра способа управления отходами \tabularnewline
\hline
CreateRoute.js & app/routes/method-type & Класс маршрута для создания вида управления отходами \tabularnewline
\hline
EditRoute.js & app/routes/method-type & Класс маршрута для редактирования вида управления отходами \tabularnewline
\hline
IndexRoute.js & app/routes/method-type & Класс маршрута для просмотра всех видов управления отходами \tabularnewline
\hline
ShowRoute.js & app/routes/method-type & Класс маршрута для просмотра вида управления отходами \tabularnewline
\hline
Noatauthorized & app/routes/notauthorized & Класс маршрута для неавторизованных пользователей \tabularnewline
\hline
NotfoundRoute & app/routes/notfound & Класс маршрута для не найденных ресурсов \tabularnewline
\hline
CreateRoute.js & app/routes/user & Класс маршрута для создания пользователя \tabularnewline
\hline
EditRoute.js & app/routes/user & Класс маршрута для редактирования пользователя \tabularnewline
\hline
IndexRoute.js & app/routes/user & Класс маршрута для просмотра всех пользователей \tabularnewline
\hline
ShowRoute.js & app/routes/user & Класс маршрута для просмотра пользователя \tabularnewline
\hline
CreateRoute.js & app/routes/waste & Класс маршрута для создания отходов \tabularnewline
\hline
EditRoute.js & app/routes/waste & Класс маршрута для редактирования отходов \tabularnewline
\hline
ShowRoute.js & app/routes/waste & Класс маршрута для просмотра отходов \tabularnewline
\hline
CreateRoute.js & app/routes/waste-type & Класс маршрута для создания вида отходов \tabularnewline
\hline
EditRoute.js & app/routes/waste-type & Класс маршрута для редактирования вида отходов \tabularnewline
\hline
IndexRoute.js & app/routes/waste-type & Класс маршрута для просмотра всех видов отходов \tabularnewline
\hline
ShowRoute.js & app/routes/waste-type & Класс маршрута для просмотра вида отходов \tabularnewline
\hline
index.js & app/routes & Общая точка входа в директорию маршрутов \tabularnewline
\hline
main.less & app/styles & Основные стили приложения \tabularnewline
\hline
sb-admin-2.less & app/styles & Стили шаблона Sb-Admin-2 \tabularnewline
\hline
utils.js & app/util & Вспомогательные функции \tabularnewline
\hline
config.js & app & Конфигурационный файл \tabularnewline
\hline
index.ejs & app & Шаблон для генерации главной html страницы приложения \tabularnewline
\hline
main.js & app & Точка входа приложения \tabularnewline
\hline
server.js & app & Вспомогательный сервер \tabularnewline
\end{longtable}

В таблице~\ref{tab:source_files_common} приведен перечень общих файлов, необходимых как для работы серверной части, так и для работы клиентской части системы.

\begin{longtable}[l]{|p{4cm}|p{5cm}|p{5.7cm}|}
\caption{Список файлов, необходимых для работы клиентской и серверной части системы}
\label{tab:source_files_common}\tabularnewline
\hline
Имя файла & Путь к файлу & Описание \tabularnewline
\endfirsthead
\hline
Имя файла & Путь к файлу & Описание \tabularnewline
\endhead
\hline
package.json & / & Конфигурационный файл npm, задающий зависимости от сторонних библиотек \tabularnewline
\hline
api.js & /bin/ & Файл, необходимый для запуска серверной части \tabularnewline
\hline
server.js & /bin/ & Файл, необходимый для запуска клиентской части \tabularnewline
\hline
dev.config.js & /webpack/ & Конфигурационный файл в режиме разработки для сборщика webpack \tabularnewline
\hline
prod.config.js & /webpack/ & Конфигурационный файл в режиме релиза для сборщика webpack \tabularnewline
\hline
webpack.config.js & /webpack/ & Общий конфигурационный файл для сборщика webpack \tabularnewline
\hline
\end{longtable}

Система использует следующие программные пакеты в качестве зависимостей:
\begin{itemize}
\item backbone (v1.3.3);
\item backbone-associations (v0.6.2);
\item backbone-react-component (v0.10.0);
\item backbone.radio (v1.0.5);
\item bcrypt (v0.8.5);
\item body-parser (v1.15.0);
\item bootstrap (v3.3.6);
\item classnames (v2.2.5);
\item connect-roles (v3.1.2);
\item cookie-parser (v1.4.1);
\item cors (v2.7.1);
\item debug (v2.2.0);
\item express (v4.13.4);
\item express-unless (v0.3.0);
\item font-awesome (v4.6.3);
\item griddle-react (v0.5.1);
\item helmet (v2.0.0);
\item i18n (v0.8.2);
\item jquery (v2.2.4);
\item jsonwebtoken (v5.7.0);
\item md5 (v2.1.0);
\item method-override (v2.3.5);
\item mongoose (v4.4.12);
\item morgan (v1.7.0);
\item nprogress (v0.2.0);
\item object-path (v0.9.2);
\item passport (v0.3.2);
\item passport-jwt (v2.0.0);
\item react (v15.0.2);
\item react-bootstrap (v0.29.4);
\item react-bootstrap-validation (v0.1.11);
\item react-dom (v15.0.2);
\item react-helmet (v3.1.0);
\item react-progress-2 (v4.2.1);
\item stardog (v0.3.1);
\item store (v1.3.20);
\item underscore (v1.8.3);
\item underscore.string (v3.3.4).
\end{itemize}

\chapter{Классы системы}

В таблице~\ref{tab:classes_server} перечислены основные классы серверной части системы с кратким их описанием.

\begin{longtable}[l]{|p{5.5cm}|p{9cm}|}
\caption{Основные классы серверной части системы}
\label{tab:classes_server}\tabularnewline
\hline
Имя класса & Описание\tabularnewline
\endfirsthead
\hline
Имя класса & Описание\tabularnewline
\endhead
\hline
Stardog & Класс надстройка над библиотекой Stardog  \tabularnewline
\hline
RdfBaseStorage & Базовый класс для всех хранилищ \tabularnewline
\hline
MethodStorage & Класс хранилище способов управления отходами \tabularnewline
\hline
SubjectStorage & Класс хранилище субъектов управления отходами \tabularnewline
\hline
WasteStorage & Класс хранилище отходов \tabularnewline
\end{longtable}

В таблице~\ref{tab:classes_client} перечислены основные классы клиентской части системы.

\begin{longtable}[l]{|p{5.5cm}|p{9cm}|}
\caption{Основные классы клиентской части системы}
\label{tab:classes_client}\tabularnewline
\hline
Имя класса & Описание\tabularnewline
\endfirsthead
\hline
Имя класса & Описание\tabularnewline
\endhead
\hline
App & Класс приложения \tabularnewline
\hline
Model & Класс модели ресурса, получаемой через REST API сервера \tabularnewline
\hline
Collection & Класс коллекции ресурсов, получаемых через REST API сервера \tabularnewline
\hline
Entity & Класс сущности ресурса \tabularnewline
\hline
ErrorHandler & Класс глобального обработчика ошибок \tabularnewline
\hline
Router & Класс роутера \tabularnewline
\hline
Route & Класс маршрута \tabularnewline
\hline
Session & Класс пользовательской сессии \tabularnewline
\hline
StateRouter & Базовый класс роутера-состояний \tabularnewline
\hline
AppContainer & Класс контейнера приложения \tabularnewline
\end{longtable}

\chapter{Испытание системы}

Испытание системы происходит на тестовом примере, поставляемым вместе с системой в виде файла онтологии учреждения ГБУЗ «Николаевское ЦРБ» в формате Turtle.

После загрузки онтологии через конфигурационный файл сервера, необходимо выполнить следующую последовательность действий в системе:
\begin{itemize}
\item авторизоваться в системе, указав свой логин и пароль;
\item перейти в профиль конкретного предприятия (в данном случае ГБУЗ Николаевское ЦРБ);
\item найти стратегию управлению отходами предприятия.
\end{itemize}

Система должна сгенерировать стратегию управления отходами предприятия, представленную в таблице~\ref{tab:test_results}.

\begin{longtable}[l]{|p{7.5cm}|p{7.5cm}|}
\caption{Сгенерированная системой стратегия управления отходами предприятия на основе тестовых данных}
\label{tab:test_results}\tabularnewline
\hline
Группа отходов & Эффективные способы обращения \tabularnewline
\endfirsthead
\hline
Группа отходов & Эффективные способы обращения \tabularnewline
\endhead
\hline
Аккумуляторы свинцовые отработанные неповрежденные, с неслитым электролитом;
Отходы упаковочной бумаги незагрязненные; Обтирочный материал, загрязненный маслами (содержание масел менее 15\%); Остатки и огарки стальных сварочных электродов; Отходы бумаги и картона от канцелярской деятельноти и делопроизводства & ООО Тора (Переработка); Николаевское городское коммунальное хозяйство (Транспортировка) \tabularnewline
\hline
Отходы гипса в кусковой форме; Электрические лампы накаливания отработанные и брак & Николаевское городское коммунальное хозяйство (Захоронение на свалке); Николаевское городское коммунальное хозяйство (Транспортировка) \tabularnewline
\hline
Отходы керамики в кусковой форме; Отходы упаковочного картона незагрязненные; Лом и отходы черных металлов с примесями или загрязненные опасными веществами (тара железная с остатками ЛКМ) & ООО Тора (Переработка); Николаевское городское коммунальное хозяйство (Транспортировка) \tabularnewline
\hline
Отходы лакокрасочных средств (кисти малярные); Ртутные лампы, люминесцентные ртутьсодержащие трубки отработанные и брак; Масла трансмисионные отработанные; Масло моторные отработанные & ООО Экоплюс (Сжигание в печи); Николаевское городское коммунальное хозяйство (Транспортировка) \tabularnewline
\hline
Отходы полиэтилена в виде пленки & Николаевское городское коммунальное хозяйство (Захоронение на свалке); Николаевское городское коммунальное хозяйство (Транспортировка) \tabularnewline
\hline
Отходы потребления на производстве, подобные коммунальным; Покрышки с металлокордом отработанные; Мусор от бытовых помещений организаций несортированный (исключая крупногабаритный); Обрезки и обрывки тканей смешанных & Николаевское городское коммунальное хозяйство (Захоронение на свалке); Николаевское городское коммунальное хозяйство (Транспортировка) \tabularnewline
\hline
Очистки овощного сырья; Пищевые отходы кухонь и организаций общественного питания; Шлак сварочный & Николаевское городское коммунальное хозяйство (Захоронение на свалке); Николаевское городское коммунальное хозяйство (Транспортировка) \tabularnewline
\hline
Количество -- 114.27 т & Стоимость -- 409588.2 р \tabularnewline
\end{longtable}

\appendixdocument{Руководство оператора}

\appendixdocument{Руководство системного программиста}

\end{document}