\documentclass[a4paper,english,russian]{G2-105}
\usepackage[T1]{fontenc}
\usepackage{listings}
\usepackage{graphicx}
\usepackage{longtable}
\usepackage{booktabs}
\usepackage{standalone}
\usepackage{newclude}
\usepackage[final]{pdfpages}
\usepackage{multirow}
\usepackage{sansmath} % Enables turning on sans-serif math mode, and using other environments
\sansmath % Enable sans-serif math for rest of document
\usepackage[math]{blindtext}

%\usepackage[utf8]{luainputenc}
\VSTUSetDocumentNumbersPrefix{}
\VSTUSetDocumentCode{ВСТАВИТЬ КОД}
\VSTUSetDocumentTypeDative{выпускной работе бакалавра}
\VSTUSetDocumentTypeGenitive{выпускную работу бакалавра}
\VSTUSetInitialData{задание, выданное научным руководителем с кафедры САПР и ПК,
утвержденное приказом ректора}

\begin{document}
\VSTUSetOrder{????–ст}{??}{??????}{201?}
\VSTUSetFaculty{Электроники и вычислительной техники}
\VSTUSetDepartment{Системы автоматизированного проектирования и ПК}
\VSTUSetDepartmentCode{??.??}
\VSTUSetDirection{??.??.?? Автоматизированные системы управления}
\VSTUSetHeadOfDepartment{Зав. кафедрой САПР и ПК}{д.т.н., ??.}{М. В. Щербаков}{Щербаков Максим Владимирович}
\VSTUSetDirector{старший преподаватель САПР и ПК}{}{А. В. Катаев}{Катаев Александр Вадимович}
\VSTUSetFacilityExpert{}{}{}{}
\VSTUSetStandardsAdviser{????}{?????}{???????????}{?????? ?????? ????????????}
\VSTUSetStudent{ИВТ-461}{Т. А. Мельников}{Мельников Тимофей Алексеевич}
\VSTUSetTitle{Портирование сверточной нейросети на ARM архитектуру с ограниченными вычислительными ресурсами и ресурсами памяти}
\VSTUSetTitleEng{Porting a convolutional neural network on an ARM architecture, taking into account the computing resources and memory resources}
%\VSTUAddChapterWordToTOC % обязательно для ПЗ в магистерских диссертациях
\VSTUInitializePZ
\abstract{Аннотация}
\par Документ представляет собой пояснительную записку к выпускной работе бакалавра на тему «Портирование сверточной нейросети на ARM архитектуру с ограниченными вычислительными ресурсами и ресурсами памяти», выполненную студентом группы ИВТ-461, Мельниковым Тимофеем Алексеевичем.
\par В данной работе рассмотрена возможность реализации алгоритмов машинного обучения, в частности прямой проход сверточной нейронной сети, на устройстве с ограниченными вычислительными ресурсами и ресурсами памяти.
\par Объём пояснительной записки составил \totalpages~страниц и включает \totalfigures~рисунков и \totaltables~таблицы. 

%\newpage
\tableofcontents
\newpage

\starchapter{Введение}
\par Задачи обработки и анализа аналоговой информации являюся одиними из самых сложных в IT-индустрии. Долгое
время такие задачи решались евристическими линейными алгоритмами, которые требовали огромных аппаратных ресурсов при малой точности результата. На протяжении последних десяти лет стремительно растет и развивается прикладная область математики цель которой изучение и развитие искусственных нейронных сетей (НС). Актуальность разработок и исследований в данной области оправдывается применением НС в различных сферах деятельности. Это автоматизация процессов анализа объектов, образов, уневерсализация управления, прогнозирование, создание экспертных систем, анализ неформализованной информации и многие другие применения. В частности, в данной дипломной работе используются нейронные сети для классификации и детектирования объектов на изображении. 
\par Наиболее существенным недостатком НС является их требовательность к вычислительным ресурсам и ресурсам памяти. Частично данная проблема решается использованием сверточных нейронных сетей, которые в виду особенностям логики работы позволяют в разы сократить потребляемые нейронной сетью ресурсы.
\par Не только искусственные нейронные сети являются трендом IT-идустрии, активно развивается коцепция интернета вещей. Диапазон встраиваемых технологий простирается от концепции умных зданий до промышленной консолидации. Интеграция встраиваемых систем и искусственных нейронных сетей позволяет автоматизировать и упростить многие процессы во многих сферах деятельности.
\par В связи с вышесказанным целью данной дипломной работы является внедрение фрейворка машинного обучения на enbedded систему C.H.I.P. и последующая его оптимизация. На основе проделанной работы необходимо сделать вывод о эффективности и рентабельности данного решения. 
\par Для достижения поставленной цели необходимо решить следующие задачи:
\begin{itemize}
\item Изучить фреймворки глубокого машинного обучения
\item Разработать консольное приложение для реализации прямого прохода нейронной сети
\item Оптимизировать использование оперативной памяти и сделать загрузку весов по мере использования
\item Разработать клиент-серверное приложение, демонстрирующее результат работы
\end{itemize}
\par В первом разделе пояснительной записки описаны фрейворки машинного обучения. Далее приведено обоснование выбора фреймворка darknet.
\par Во втором разделе описаны используемые модели нейронных сетей и алгоритм прямого прохода.
\par Третей раздел посвящен разворачиванию фреймворка на устройстве C.H.I.P. и оптимизации работы алгоритма прямого прохода. Так же описана разработка клиент-серверной части для визуализации работы приложения. 
\newpage

\chapter{Обзор фреймворков машинного обучения}



\chapter{Используемые алгоритмы и модели}
\section{Теоретические основы нейронных сетей}
\subsection{Перцептрон - основа нейронных сетей}
\par В основе современной концепции 

\chapter{Проектирование системы}


\newpage
\starchapter{Заключение}
\newpage
\begin{thebibliography}{1}
    \bibitem{1} http://www.raai.org/library/books/mcculloch/mcculloch.pdf
\end{thebibliography}

\appendixdocument{Техническое задание}
\end{document}